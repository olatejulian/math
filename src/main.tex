\documentclass[
    12pt, % font size
    a4paper, % paper size
]{book}

\usepackage{amsfonts, amsmath, amssymb, amsthm}
\usepackage[portuguese]{babel}
\usepackage[utf8]{inputenc}
\usepackage{subfiles}

\bibliographystyle{apalike}

\title{Estudos Matemáticos}
\author{Julian Lucas Faria Olate}

\newtheorem{defi}{Definição}[section]

\newcommand{\Set}[1]{\mathbb{#1}}

\begin{document}
    \maketitle
    \tableofcontents

    \chapter{Conceitos Matemáticos}\label{ch:conceitos-matemáticos}
        \section{Definições}\label{sec:definicoes}
            Esta seção é baseada nas definições dadas por \cite{dos-santos2014}:

            \begin{defi}[Axioma]\label{defi:axioma}
                É uma proposição considerada verdadeira, não demonstrável, geralmente é utilizado como princípio na construção de uma teoria ou base de argumentação.
            \end{defi}

            \begin{defi}[Axiomático]\label{defi:axiomatico}
                É algo evidente, inquestionável,\newline incontestável, é relativo aos axiomas.
            \end{defi}

            \begin{defi}[Sistema Axiomático]\label{defi:sistema-axiomatico}
                É o conjunto dos axiomas que definem uma teoria, constituem as verdades mais simples a partir das quais se demonstram os resultados.
            \end{defi}

            \begin{defi}[Teorema]\label{defi:teorema}
                É uma proposição que pode ser demonstrada de maneira lógica a partir de uma axioma ou de outros teoremas que tenham sido previamente demonstrados.
            \end{defi}

            \begin{defi}[Conjunto]\label{defi:conjunto}
                É um agregado ou coleção de elementos ou pontos que pode ser definido por uma quantidade de propriedades distintas.
            \end{defi}

            \begin{defi}[Operação Matemática]\label{defi:operacao-matematica}
                É qualquer tipo de procedimento que envolve uma quantidade de elementos de um conjunto e resulta em um terceiro elemento de um conjunto.
            \end{defi}

            \begin{defi}[Operação Binária]\label{defi:operacao-binaria}
                É uma regra que associa dois elementos de um conjunto a um terceiro elemento de um conjunto.
            \end{defi}

            \begin{defi}[Estrutura Algébrica]\label{defi:estrutura-algebrica}
                É um modelo associado a um conjunto de elementos e uma ou mais operações binárias que satisfazem um sistema axiomático.
            \end{defi}

            \begin{defi}[Par Ordenado]\label{defi:par-ordenado}
                É um par de elementos a qual a ordem de ocorrência de cada elemento é significante. Geralmente representado por $(a,b)$. Seja $a$ e $b$ elementos quaisquer.
            \end{defi}

            \begin{defi}[Grupo]\label{defi:grupo}
                Seja $\Set{G}$ um conjunto de elementos e $\*$ uma operação binária em $\Set{G}$. O par ordenado $(\Set{G},*)$ é definido um grupo se são satisfeitos os seguintes axiomas:

                \begin{enumerate}
                    \item \textbf{Fechamento:} Para toda operação binária $a*b=c$, se $a,b \in\Set{G}$, então $c \in\Set{G}$.

                    \item \textbf{Associatividade:} Para quaisquer elementos $a,b,c \in\Set{G}$, $a*(b*c)=(a*b)*c$

                    \item \textbf{Existência do elemento neutro:} Existe um elemento $\Set{I} \in\Set{G}$ tal que $a*\Set{I}=\Set{I}*a=a$ para qualquer elemento $a \in\Set{G}$

                    \item \textbf{Existência do elemento inverso:} Dado o elemento neutro $\Set{I} \in\Set{G}$, Existe um elemento $a^{-1} \in\Set{G}$ tal que $a*a^{-1}=a^{-1}*a=\Set{I}$ para qualquer elemento $a \in\Set{G}$
                \end{enumerate}
            \end{defi}

            \begin{defi}[Anel]\label{defi:anel}
                É uma estrutura algébrica que consiste em uma tripla ordenada $(A,+,\cdot)$, seja $A$ um conjunto que possui o elemento vazio $\Set{0}$ e duas operações binárias definidas em $A$, $+$ e $\cdot$ que satisfazem os seguintes axiomas:

                \begin{enumerate}
                    \item \textbf{Associatividade de $+$:} $(\forall a,b,c \in A): (a+b)+c=a+(b+c)$

                    \item \textbf{Comutatividade de $+$:} $(\forall a,b \in A): a+b=b+a$

                    \item \textbf{Existência do elemento neutro de $+$:} $(\forall a \in A): \Set{I}+a=a+\Set{I}=a$

                    \item \textbf{Existência do elemento inverso de $+$:} $(\forall a \in A): a+a^{-1}=a^{-1}+a=\Set{I}$
                    \item \textbf{Associatividade de $\cdot$:} $(\forall a,b,c \in A): (a\cdot b)\cdot c=a\cdot (b\cdot c)$

                    \item \textbf{Distributividade de $\cdot$ em relação a $+$ (à esquerda e à direita):} $(\forall a,b,c \in A): a\cdot (b+c)=a\cdot b+a\cdot c$ e $(\forall a,b,c \in A): a \cdot (b+c)=a \cdot b+a \cdot c \wedge (a+b)\cdot c=a\cdot c+b\cdot c$
                \end{enumerate}
            \end{defi}

            \begin{defi}[Anel Comutativo]\label{defi:anel-comutativo}
                Um anel $(A,+,\cdot)$ é denominado anel comutativo se além de satisfazer os axiomas de um anel, também satisfaz o axioma de comutatividade para a operação $\cdot$, sendo $(\forall a,b \in A): a\cdot b=b\cdot a$.
            \end{defi}

            \begin{defi}[Corpo]\label{defi:corpo}
                Um anel comutativo $F$ é denominado corpo se satisfaz a condição: $(\forall a \in F\ \backslash\ \{0\})(\exists! b \in F): a\cdot b=\Set{I}$
            \end{defi}

            \begin{defi}[Espaço Vetorial]\label{defi:espaco-vetorial}
                Seja $V$ um conjunto sobre o corpo $K$ é denominado espaço vetorial se é munido de de duas operações binárias, $+$ e $.$ que satisfazem os seguintes axiomas:

                \begin{enumerate}
                    \item \textbf{Associatividade de $+$:} $(\forall a,b,c \in V): (a+b)+c=a+(b+c)$

                    \item \textbf{Comutatividade de $+$:} $(\forall a,b \in V): a+b=b+a$

                    \item \textbf{Existência do elemento neutro de $+$:} $(\forall a \in V): \Set{I}+a=a+\Set{I}=a$

                    \item \textbf{Existência do elemento inverso de $+$:} $(\forall a \in V): a+a^{-1}=a^{-1}+a=\Set{I}$


                \end{enumerate}
            \end{defi}

    \bibliography{bibliography}
\end{document}
