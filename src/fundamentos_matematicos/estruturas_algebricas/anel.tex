\documentclass[../../main]{subfiles}

\begin{document}
    \begin{defi}[Anel]\label{defi:anel}
        É uma estrutura algébrica que consiste em uma tripla ordenada $(A,+,\cdot)$, seja $A$ um conjunto que possui o elemento vazio $\Set{0}$ e duas operações binárias definidas em $A$, $+$ e $\cdot$ que satisfazem os seguintes axiomas:

        \begin{enumerate}
            \item \textbf{Associatividade de $+$:} $(\forall a,b,c \in A): (a+b)+c=a+(b+c)$

            \item \textbf{Comutatividade de $+$:} $(\forall a,b \in A): a+b=b+a$

            \item \textbf{Existência do elemento neutro de $+$:} $(\forall a \in A): \Set{I}+a=a+\Set{I}=a$

            \item \textbf{Existência do elemento inverso de $+$:} $(\forall a \in A): a+a^{-1}=a^{-1}+a=\Set{I}$
            \item \textbf{Associatividade de $\cdot$:} $(\forall a,b,c \in A): (a\cdot b)\cdot c=a\cdot (b\cdot c)$

            \item \textbf{Distributividade de $\cdot$ em relação a $+$ (à esquerda e à direita):} $(\forall a,b,c \in A): a\cdot (b+c)=a\cdot b+a\cdot c$ e $(\forall a,b,c \in A): a \cdot (b+c)=a \cdot b+a \cdot c \wedge (a+b)\cdot c=a\cdot c+b\cdot c$
        \end{enumerate}
    \end{defi}

    \begin{defi}[Anel Comutativo]\label{defi:anel-comutativo}
        Um anel $(A,+,\cdot)$ \ref{defi:anel} é denominado anel comutativo se além de satisfazer os axiomas de um anel, também satisfaz o axioma de comutatividade para a operação $\cdot$, sendo $(\forall a,b \in A): a\cdot b=b\cdot a$.
    \end{defi}
\end{document}
