\documentclass[../../main]{subfiles}

\begin{document}
    \begin{defi}[Espaço Vetorial]\label{defi:espaco-vetorial}
        Se $V$ é dito ser um espaço vetorial sobre o corpo $K$ \ref{defi:corpo} se e somente se é munido de de duas operações binárias, $+$ e $.$ que satisfazem os seguintes axiomas:


        \begin{enumerate}
            \item \textbf{Associatividade de $+$:} $(\forall a,b,c \in V): (a+b)+c=a+(b+c)$

            \item \textbf{Comutatividade de $+$:} $(\forall a,b \in V): a+b=b+a$

            \item \textbf{Existência do elemento neutro de $+$:} $(\forall a \in V, \exists! \Set{I} \in V): \Set{I}+a=a+\Set{I}=a$

            \item \textbf{Existência do elemento inverso de $+$:} $(\forall a \in V, \exists! a^{-1} \in V): a+a^{-1}=a^{-1}+a=\Set{I}$

            \item \textbf{Associatividade de $.$:} $(\forall a \in V, \alpha, \beta \in K): \alpha(\beta.a) = (\alpha.\beta).a $

            \item \textbf{Distributividade de $.$ em relação a $+$:} $(\forall a, b \in V, \alpha in K): \alpha.(a+b)=\alpha.a+\alpha.b$

            \item \textbf{Distributividade de $.$ em relação a soma de escalares:} $(\forall a \in V, \alpha, \beta \in K): (\alpha+\beta).a=(\alpha.a)+(\beta.a)$

            \item \textbf{Existência do elemento neutro de $.$:} $(\forall a \in V, \Set{I} \in K): \alpha.a=a$
        \end{enumerate}
    \end{defi}

    A definição \ref{defi:espaco-vetorial} foi baseada em \cite{lima-2012}.
\end{document}
